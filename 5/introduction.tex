\IEEEPARstart{E}{thernet} over twisted pair remains the default physical medium
in campus labs and home networks because it is cheap, robust, and easy to
terminate in the field. In this short lab we built and tested our own Cat5e
patch cable with RJ45 connectors to understand the practical details that
underpin the link layer: pair ordering, crimping, and continuity testing. The
goal of this report is to document the assembly process, highlight common
pitfalls, and relate the observed results to the standards that govern copper
cabling.

\subsection{Theoretical Framework}
\subsubsection{Ethernet over Twisted Pair}
IEEE~802.3 defines Ethernet physical layers that run over balanced twisted
pairs (Cat5e/Cat6). Each pair is tightly twisted to reduce crosstalk, and
100BASE-TX/1000BASE-T use differential signaling and MLT-3 or PAM-5 line codes
to transmit data at high speed while controlling EMI~\cite{ieee8023}. Link
integrity depends on correct pair terminations and consistent impedance along
the cable.

\subsubsection{RJ45 Pinout and T568A}
TIA/EIA-568-B specifies two wiring schemes (T568A and an alternate variant) for
terminating four twisted pairs on an 8P8C (RJ45) connector~\cite{tia568b}. Each
scheme preserves pairing (1-2, 3-6, 4-5, 7-8) while assigning colors: in T568A
the order from pin~1 to~8 is white/green, green, white/orange, blue,
white/blue, orange, white/brown, brown. Using the same scheme on both ends
produces a straight-through patch cable.

\subsubsection{Crimping and Continuity}
Crimping drives the connector's metal blades through the insulation to make
solid contact with each conductor. Proper termination requires even wire
lengths, the jacket inside the plug for strain relief, and fully seated pins.
Continuity testers sequence voltage across each pin to verify that 1$\rightarrow$1,
2$\rightarrow$2, \dots, 8$\rightarrow$8 are intact and that no shorts or
crossed pairs exist~\cite{fluketest}.

\subsection{State of the Art}
Field studies show that poor terminations and pair misorders are a leading
cause of Ethernet link flaps and reduced throughput, especially when upgrading
legacy cabling to gigabit speeds~\cite{biegel2006}. Industry best practices
recommend sticking to a single standard (here, T568A), maintaining twist all
the way to the plug, and validating with calibrated testers to avoid hidden
faults that only surface under load~\cite{fluketest}. Recent work on higher
speeds over copper (2.5G/5GBASE-T) further tightens requirements on return loss
and crosstalk, making correct pinouts and clean crimps even more critical
~\cite{ieee8023bz}.
