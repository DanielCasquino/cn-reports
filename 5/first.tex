\subsection{Ethernet Cabling Experience}

\subsubsection{Preparing the Cable}
We began by cutting a 1~meter Cat5e segment and stripping about 4~cm of the
outer jacket on each end. After exposing the four twisted pairs, we straightened
them to make alignment easier (Figure~\ref{fig:stripping}). The color pairs
appeared as expected: white/orange--orange, white/green--green,
white/blue--blue, and white/brown--brown.

\begin{figure}[htbp]
	\centering
	\includegraphics[width=0.9\linewidth]{img/1/pelando.jpeg}
\caption{Stripping the jacket and exposing the pairs}\label{fig:stripping}
\end{figure}

\subsubsection{Ordering and Trimming}
Following the lab rule (even-numbered group), we used the T568A scheme on both ends.
We aligned the conductors from pin~1 to pin~8 as white/green, green,
white/orange, blue, white/blue, orange, white/brown, brown
(Figure~\ref{fig:pairs_aligned}), verified that the twists stayed close to the plug,
and trimmed the eight wires to a uniform length (Figure~\ref{fig:measuring1}).
Keeping the jacket inside the connector mouth helps
with strain relief once crimped.

\begin{figure}[htbp]
	\centering
	\includegraphics[width=0.9\linewidth]{img/1/internos.jpeg}
	\caption{Pairs aligned (T568A order)}\label{fig:pairs_aligned}
\end{figure}

\begin{figure}[htbp]
	\centering
	\includegraphics[width=0.9\linewidth]{img/1/midiendo1.jpeg}
\caption{Equalizing jacket lengths before trimming}\label{fig:measuring1}
\end{figure}

\subsubsection{Crimping}
With the wires ordered and trimmed, we inserted them into the RJ45 plug (clip
facing down) and ensured each conductor reached the front of the connector.
Then we crimped firmly so the metal blades pierced the insulation and the plug
latched onto the jacket (Figure~\ref{fig:cabezal}). We repeated the process on
the other end, again keeping the T568A order to form a straight-through patch.

\begin{figure}[htbp]
	\centering
	\includegraphics[width=0.9\linewidth]{img/1/cabezal.jpeg}
	\caption{Left: crimped plug; right: uncrimped plug}\label{fig:cabezal}
\end{figure}

\subsubsection{Testing}
Finally, we connected both ends to the cable tester. All eight LEDs advanced in
sequence on both units, indicating correct pin-to-pin continuity and no crossed
pairs (Figure~\ref{fig:test}). We also recorded short and long tester videos to
capture the live blinking pattern (test videos: \href{https://drive.google.com/drive/folders/1X3n0K6UxmNHpi8Wi7MtSqXyyAPwwePR3?usp=sharing}{Google Drive}).

\begin{figure}[htbp]
	\centering
	\includegraphics[width=0.9\linewidth]{img/1/test.jpeg}
	\caption{Continuity test with the handheld tester}\label{fig:test}
\end{figure}

\subsubsection{Result}
The finished cable is shown in Figure~\ref{fig:terminado}. With the tester
passing and strain relief in place, the patch is ready for everyday Ethernet
use at 100/1000~Mbps.

\begin{figure}[htbp]
	\centering
	\includegraphics[width=0.9\linewidth]{img/1/terminado.jpeg}
	\caption{Finished cable, ready to use}\label{fig:terminado}
\end{figure}
