\subsection{Ethernet Cabling Experience}

\subsubsection{Preparing the Cable}
We began by cutting a 1~meter Cat5e segment and stripping about 4~cm of the
outer jacket on each end. After exposing the four twisted pairs, we straightened
them to make alignment easier (Figure~\ref{fig:pelando}). The color pairs
appeared as expected: white/orange--orange, white/green--green,
white/blue--blue, and white/brown--brown.

\begin{figure}[htbp]
    \centering
    \includegraphics[width=0.9\linewidth]{img/1/pelando.jpeg}
    \caption{Desnudando la chaqueta y separando pares}\label{fig:pelando}
\end{figure}

\subsubsection{Ordering and Trimming}
Following the lab rule (grupo impar), we used the T568B scheme on both ends.
We aligned the conductors from pin~1 to pin~8 as white/orange, orange,
white/green, blue, white/blue, green, white/brown, brown
(Figure~\ref{fig:internos}), verified that the twists stayed close to the plug,
and trimmed the eight wires to a uniform length (Figures~\ref{fig:midiendo1}
and~\ref{fig:midiendo2}). Keeping the jacket inside the connector mouth helps
with strain relief once crimped.

\begin{figure}[htbp]
    \centering
    \includegraphics[width=0.9\linewidth]{img/1/internos.jpeg}
    \caption{Pares alineados según T568B}\label{fig:internos}
\end{figure}

\begin{figure}[htbp]
    \centering
    \includegraphics[width=0.9\linewidth]{img/1/midiendo1.jpeg}
    \caption{Nivelando longitudes antes del corte}\label{fig:midiendo1}
\end{figure}

\begin{figure}[htbp]
    \centering
    \includegraphics[width=0.9\linewidth]{img/1/midiendo2.jpeg}
    \caption{Conectores listos para crimpado}\label{fig:midiendo2}
\end{figure}

\subsubsection{Crimping}
With the wires ordered and trimmed, we inserted them into the RJ45 plug (clip
facing down) and ensured each conductor reached the front of the connector.
Then we crimped firmly so the metal blades pierced the insulation and the plug
latched onto the jacket (Figure~\ref{fig:cabezal}). We repeated the process on
the other end, again keeping the T568B order to form a straight-through patch.

\begin{figure}[htbp]
    \centering
    \includegraphics[width=0.9\linewidth]{img/1/cabezal.jpeg}
    \caption{Conector RJ45 crimpado con T568B}\label{fig:cabezal}
\end{figure}

\subsubsection{Testing}
Finally, we connected both ends to the cable tester. All eight LEDs advanced in
sequence on both units, indicating correct pin-to-pin continuity and no crossed
pairs (Figure~\ref{fig:test}). We also recorded short and long tester videos to
capture the live blinking pattern (files \texttt{img/1/test-corto.mp4} y
\texttt{img/1/test-largo.mp4}).

\begin{figure}[htbp]
    \centering
    \includegraphics[width=0.9\linewidth]{img/1/test.jpeg}
    \caption{Prueba de continuidad con el tester}\label{fig:test}
\end{figure}

\subsubsection{Result}
The finished cable is shown in Figure~\ref{fig:terminado}. With the tester
passing and strain relief in place, the patch is ready for everyday Ethernet
use at 100/1000~Mbps.

\begin{figure}[htbp]
    \centering
    \includegraphics[width=0.9\linewidth]{img/1/terminado.jpeg}
    \caption{Cable terminado, listo para usar}\label{fig:terminado}
\end{figure}
