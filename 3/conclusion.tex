\section{Conclusions and Recommendations}
\subsection{First Experience}
\textbf{Conclusion:} We saw how UDP packets work and what info they carry.
Comparing with TCP, it's clear TCP adds more reliability with handshakes and
acknowledgments.

\textbf{Recommendation:} Use UDP for speed when you don't need reliability. For
anything important, stick with TCP.

\subsection{Second Experience}
\textbf{Conclusion:} Coding UDP in Python was simple and fast. We learned about
packet structure and saw how using the wrong IP or network setup can break
things.

\textbf{Recommendation:} Double-check your network setup and IPs when using UDP.
Remember, UDP doesn't guarantee delivery.

\subsection{Third Experience}
\textbf{Conclusion:} Capturing a TCP file transfer helped us understand the
handshake and how packets are sequenced. Ports and sequence numbers matter a
lot.

\textbf{Recommendation:} If you're having connection issues, check the handshake
and sequence numbers with a tool like Wireshark.

\subsection{Fourth Experience}
\textbf{Conclusion:} Looking at TCP's sequence numbers and congestion control
showed how it ramps up speed and handles network traffic smartly.

\textbf{Recommendation:} Watch how TCP manages traffic, especially for big
transfers. Make sure your network can handle it.

\subsection{Fifth Experience}
\textbf{Conclusion:} Building a reliable data transfer protocol in Python was a
good way to see why checks, ACKs, and timeouts matter. Simulating errors made
it clear how tricky real networks can be.

\textbf{Recommendation:} Always add error checking and recovery to your
protocols. Test with simulated problems to make sure it works.

\subsection{Sixth Experience}
\textbf{Conclusion:} Our TCP server-client file transfer worked smoothly on the
local network. TCP handled everything, and there were no lost packets.

\textbf{Recommendation:} For sending files, TCP is the way to go. Use Wireshark
to check things if you run into trouble.

% if have a single appendix:
%\appendix[Proof of the Zonklar Equations]
% or
%\appendix  % for no appendix heading
% do not use \section anymore after \appendix, only \section*
% is possibly needed

% use appendices with more than one appendix
% then use \section to start each appendix
% you must declare a \section before using any
% \subsection or using \label (\appendices by itself
% starts a section numbered zero.)
%
