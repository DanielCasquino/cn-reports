\subsection{Report Research}

This subsection answers the report research questions, and provides additional
information that would otherwise clutter the main report.

\subsubsection{Risks of using UDP}
% What are the potential risks or vulnerabilities of using UDP in a network?
% Did you observe any unusual or potentially malicious UDP traffic?
Because UDP offers speed and low overhead, it can be used to send large numbers
of packets quickly, which can overwhelm servers without sufficient resources or
control for this types of situtations. This makes UDP an easy choice for DDoS
attacks, where attackers send large volumes of packets, and UDP's lack of
handshakes offer no limitation on the sender side~\cite{mirkovic2004taxonomy}.
During our experiences, we did not observe any particularly suspicious UDP
traffic, but the risk still remains.

\subsubsection{Detecting suspicious UDP traffic with Wireshark}
% How could an administrator use Wireshark to detect suspicious UDP
% behaviour, such as a port scan or DDoS attemp?
There are several types of port scanning techniques, and as such, different
ways to detect them. For this particular example, we'll analyze the classic SYN
scan, which sends SYN packets to several ports on the target computer. If the
port is open, the sender will receive a SYN-ACK, while close ports will usually
send a RST packet~\cite{rfc3360}. \\

Using Wireshark, an admin could filter SYN packets, and check if they were sent
to multiple ports in a short time frame. They could also check for high RST
packet frequency, which could indicate that several closed ports were
scanned.\\

In the case of a DDoS attempt, an admin could filter UDP packets by IP, and
check for a single repeating source IP.\@If the attack comes from multiple
sources, however, this task becomes more difficult.

\subsubsection{Testing TCP congestion mechanisms}
% abrir netflix, yt, etc
% plot wireshark capture and observe the window behaviour and tcp
% sequencing behaviour
% is there any packet loss?