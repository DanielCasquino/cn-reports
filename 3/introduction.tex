% The very first letter is a 2 line initial drop letter followed
% by the rest of the first word in caps.
% 
% form to use if the first word consists of a single letter:
% \IEEEPARstart{A}{demo} file is ....
% 
% form to use if you need the single drop letter followed by
% normal text (unknown if ever used by the IEEE):
% \IEEEPARstart{A}{}demo file is ....
% 
% Some journals put the first two words in caps:
% \IEEEPARstart{T}{his demo} file is ....
% 
% Here we have the typical use of a "T" for an initial drop letter
% and "HIS" in caps to complete the first word.
\IEEEPARstart{I}{P} addresses allow us to identify devices on a network,
and allow us to communicate with the precise device we want to reach.
However, there's only so many IP addresses that can be assigned and/or remembered, and as such,
a scalable solution is needed.

In these lab sessions, we experimented with the Domain Name System (DNS), which
functions as a system that maps strings of numbers to human-readable hostnames.
The goal of this report is to document and summarize our findings and
conclusions.
% You must have at least 2 lines in the paragraph with the drop letter
% (should never be an issue)

%\hfill mds

%\hfill August 26, 2015

\section{Theoretical Framework}

\subsection{User Datagram Protocol (UDP)}
UDP provides a connectionless, best-effort datagram service over IP: no delivery, ordering, or duplication guarantees, and no built-in congestion control. Its minimal header and lack of handshakes yield low latency and overhead, which suits interactive or real-time apps (VoIP, streaming, games) and DNS queries where timeliness outweighs perfect reliability \cite{rfc768,kurose:topdown8e}. Any reliability, pacing, or congestion response must be implemented by the application layer.

\subsection{Reliable Data Transfer (RDT)}
RDT is a design pattern for building reliability atop an unreliable channel using sequence numbers, ACK/NAK, timeouts and retransmissions (ARQ). Stop-and-Wait is simplest; pipelined variants (Go-Back-N, Selective Repeat) improve utilization. In high BDP paths, hybrid ARQ+FEC can preserve throughput while lowering delay and receiver buffering compared to pure ARQ at large windows \cite{kurose:topdown8e,ghaderi2013scalablerdt}.

\subsection{Transmission Control Protocol (TCP)}
TCP offers a reliable, in-order byte stream with flow control (rwnd) and congestion control. It tracks bytes with sequence/ACK numbers, retransmits losses, and uses handshakes and options (e.g., MSS, SACK) \cite{rfc793,kurose:topdown8e}. Classic loss-based control (Reno/CUBIC) can induce bufferbloat or underuse lossy links; newer approaches (e.g., BBR) model bottleneck bandwidth and RTT to operate with high throughput and low queueing delay \cite{cardwell2017bbr}.

\subsection{Port Scanning}
Port scanning probes TCP/UDP ports to infer open/closed/filtered services by analyzing protocol-specific responses (e.g., SYN$\rightarrow$SYN-ACK vs. RST, ICMP Port Unreachable for UDP). It is essential for asset discovery and security auditing, yet aggressive scans are detectable; defenders counter with filtering, rate-limits and tarpits \cite{lyon:nmap}.

\subsection{Distributed Denial-of-Service (DDoS)}
DDoS floods exhaust bandwidth, state or CPU by coordinating many sources. Variants include SYN floods, UDP floods, application-layer floods, and reflection/amplification via open UDP services. Mitigation blends network filtering, rate-limiting, anycast/CDN scrubbing, and disabling misconfigured amplifiers \cite{mirkovic2004taxonomy}.

% An example of a floating figure using the graphicx package.
% Note that \label must occur AFTER (or within) \caption.
% For figures, \caption should occur after the \includegraphics.
% Note that IEEEtran v1.7 and later has special internal code that
% is designed to preserve the operation of \label within \caption
% even when the captionsoff option is in effect. However, because
% of issues like this, it may be the safest practice to put all your
% \label just after \caption rather than within \caption{}.
%
% Reminder: the "draftcls" or "draftclsnofoot", not "draft", class
% option should be used if it is desired that the figures are to be
% displayed while in draft mode.
%
%\begin{figure}[!t]
%\centering
%\includegraphics[width=2.5in]{myfigure}
% where an .eps filename suffix will be assumed under latex, 
% and a .pdf suffix will be assumed for pdflatex; or what has been declared
% via \DeclareGraphicsExtensions.
%\caption{Simulation results for the network.}
%\label{fig_sim}
%\end{figure}

% Note that the IEEE typically puts floats only at the top, even when this
% results in a large percentage of a column being occupied by floats.

% An example of a double column floating figure using two subfigures.
% (The subfig.sty package must be loaded for this to work.)
% The subfigure \label commands are set within each subfloat command,
% and the \label for the overall figure must come after \caption.
% \hfil is used as a separator to get equal spacing.
% Watch out that the combined width of all the subfigures on a 
% line do not exceed the text width or a line break will occur.
%
%\begin{figure*}[!t]
%\centering
%\subfloat[Case I]{\includegraphics[width=2.5in]{box}%
%\label{fig_first_case}}
%\hfil
%\subfloat[Case II]{\includegraphics[width=2.5in]{box}%
%\label{fig_second_case}}
%\caption{Simulation results for the network.}
%\label{fig_sim}
%\end{figure*}
%
% Note that often IEEE papers with subfigures do not employ subfigure
% captions (using the optional argument to \subfloat[]), but instead will
% reference/describe all of them (a), (b), etc., within the main caption.
% Be aware that for subfig.sty to generate the (a), (b), etc., subfigure
% labels, the optional argument to \subfloat must be present. If a
% subcaption is not desired, just leave its contents blank,
% e.g., \subfloat[].

% An example of a floating table. Note that, for IEEE style tables, the
% \caption command should come BEFORE the table and, given that table
% captions serve much like titles, are usually capitalized except for words
% such as a, an, and, as, at, but, by, for, in, nor, of, on, or, the, to
% and up, which are usually not capitalized unless they are the first or
% last word of the caption. Table text will default to \footnotesize as
% the IEEE normally uses this smaller font for tables.
% The \label must come after \caption as always.
%
%\begin{table}[!t]
%% increase table row spacing, adjust to taste
%\renewcommand{\arraystretch}{1.3}
% if using array.sty, it might be a good idea to tweak the value of
% \extrarowheight as needed to properly center the text within the cells
%\caption{An Example of a Table}
%\label{table_example}
%\centering
%% Some packages, such as MDW tools, offer better commands for making tables
%% than the plain LaTeX2e tabular which is used here.
%\begin{tabular}{|c||c|}
%\hline
%One & Two\\
%\hline
%Three & Four\\
%\hline
%\end{tabular}
%\end{table}

% Note that the IEEE does not put floats in the very first column
% - or typically anywhere on the first page for that matter. Also,
% in-text middle ("here") positioning is typically not used, but it
% is allowed and encouraged for Computer Society conferences (but
% not Computer Society journals). Most IEEE journals/conferences use
% top floats exclusively. 
% Note that, LaTeX2e, unlike IEEE journals/conferences, places
% footnotes above bottom floats. This can be corrected via the
% \fnbelowfloat command of the stfloats package.

\section{State of the Art}

\subsection{RDT Beyond ARQ}
For high-speed or lossy environments, coding augments ARQ to cut delay and receiver buffering while sustaining full throughput; analytical results show FEC+ARQ can be asymptotically more delay- and buffer-efficient than selective repeat under large windows \cite{ghaderi2013scalablerdt}.

\subsection{UDP in Modern Stacks}
UDP underpins classic real-time media and DNS, and now serves as substrate for QUIC/HTTP-3. QUIC brings reliable, encrypted, multiplexed streams in user space, avoiding TCP’s head-of-line blocking and reducing handshake costs; it is standardized by the IETF \cite{rfc9000}.

\subsection{TCP Evolution}
Beyond SACK/PRR/CUBIC, \textit{BBR} shifts from loss-driven to model-based control, probing for bottleneck bandwidth and min-RTT to keep queues shallow and goodput high; deployments report lower latency and higher throughput across diverse paths \cite{cardwell2017bbr}.



\section{Methodology}
We used a capture–execute–analyze loop: (1) configure the scenario (client–server scripts or tool-driven probes), (2) generate traffic, capturing with Wireshark, and (3) analyze traces with display filters and flow/sequence graphs. Python sockets produced UDP/TCP traffic (and controlled loss/bursts); standard tools (\texttt{nslookup}, \texttt{ping}, \texttt{nmap}) exercised name resolution and scanning. For transport behavior we inspected 3-way handshakes, seq/ack dynamics, SACK/dupACKs, windows, RTTs and retransmissions; for UDP we verified ports, checksums, and server/client behavior. We documented steps and sketched simple block/flow diagrams to keep experiments repeatable.

