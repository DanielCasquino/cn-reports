\subsection{Fifth Experience}

In this experience, we implemented the Reliable Data Transfer (RDT) protocol
using Python. We used two scripts (sender and receiver) to simulate
communication over an unreliable channel, using ACK/NAK, a stop-and-wait
protocol with timeouts, and randomized packet corruption.

To avoid cluttering the report with code snippets, the full implementation (and
receiver logs) can be found in the GitHub repository linked in the introduction
section.

The sender script first defines an array of messages. It then iterates over
them, sending each one with their corresponding sequence number. Before
sending, however, there is a small chance of packet corruption.

The receiver listens for incoming packets, and uses the provided checksum to
verify packet integrity. If the packet is valid and has a correct sequence
number, it sends and ACK. Otherwise, the sender receives a NAK and retransmits
the packet.

Lastly, the sender implements a timeout mechanis, in case ACK/NAK responses
take longer than expected. The timeout, IPs, ports, and corruption parameters
can all be configured when creating an instance of the sender and receiver
classes.

% TODO: redactar