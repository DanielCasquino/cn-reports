
\section*{Conclusions and Recommendations}

The experiments in this lab consolidated the intuition gained during the
previous reports: traceroute gave us a concrete feel for TTL depletion and the
mechanics of IPv4 fragmentation, the IPv6 AAAA capture proved that the stack is
already dual-stack on campus, the NAT traces showed that only the addresses and
checksums change as traffic crosses a boundary, ICMP error captures explained
why type/code pairs carry so much diagnostic value, the PMTUD sweeps clarified
how routers signal ``fragmentation needed,'' and the DHCP handshake reminded
us that leases, options, and transaction IDs need to line up before a host
joins the network. Altogether, the network layer stopped feeling like a set of
abstract fields and became something we can inspect packet by packet.

Going forward we plan to always keep the console output and Wireshark capture
side by side when presenting checkpoints, capture both sides whenever we
analyze a middlebox, record real IPv6 traffic to complement IPv4-only
scenarios, document parameter-request lists and transaction IDs for DHCP-like
protocols, and repeat PMTUD probes with different payload sizes until we find
the usable MTU. These habits should make future reports more cohesive and cut
down on last-minute evidence gathering.
% if have a single appendix:
%\appendix[Proof of the Zonklar Equations]
% or
%\appendix  % for no appendix heading
% do not use \section anymore after \appendix, only \section*
% is possibly needed

% use appendices with more than one appendix
% then use \section to start each appendix
% you must declare a \section before using any
% \subsection or using \label (\appendices by itself
% starts a section numbered zero.)
%
