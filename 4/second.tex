\subsection{Second Experience}

\subsubsection{IPv6}
To generate actual IPv6 traffic we sent an explicit AAAA query with
\texttt{dig @2606:4700:4700::1111 AAAA youtube.com} while capturing on the
wireless interface. The request used our temporary global address
\texttt{2800:4b0:4031:48a4:b1ed:7cdc:87cb:6344} as source and Cloudflare's
resolver \texttt{2606:4700:4700::1111} as destination. Wireshark reports a flow
label of \texttt{0x9aa5e}, a payload length of 60 bytes (UDP header plus DNS
query), Hop Limit 64, and Next Header 17, so the IPv6 datagram is delivered to
UDP. The DNS header confirms that it is a standard recursive query with one
question and the EDNS0 cookie option.

The response datagram mirrors those fields: the Hop Limit was 57 when it
arrived, the flow label changed to \texttt{0xbbe04} (because the resolver can
choose its own identifier), and the payload length increased to 76 bytes to
accommodate the AAAA answer and the OPT record. Only one IPv6 address was
returned for \texttt{youtube.com}, namely
\texttt{2800:3f0:4005:418::200e}, which matches the addresses resolved via
regular DNS on the same network. Figure~\ref{fig:exp2_ipv6_flow} is the pending
Wireshark evidence for the pair of packets.

\begin{figure}[htbp]
    \centering
    \fbox{\parbox{0.9\linewidth}{\centering Placeholder --- IPv6 DNS AAAA request/response capture.}}
    \caption{IPv6 AAAA exchange (pending evidence)}\label{fig:exp2_ipv6_flow}
\end{figure}

\subsubsection{Practical Exercise}

Theoretical questions:

\begin{itemize}
    \item 10.15.32.200/8: class A, NA 10.0.0.0, BA 10.255.255.255
    \item 172.20.45.7/12: class B, NA 172.16.0.0, BA 172.16.255.255
    \item 192.168.10.25/16: class C, NA 192.168.0.0, BA 192.168.255.255
\end{itemize}

After choosing the first TCP packet, we expanded and examined the IPv4 section.

\begin{figure}[htbp]
    \centering
    \includegraphics[width=1\linewidth]{img/2/6_1.png}
    \caption{TCP packet's IPv4 section}\label{fig:exp2_6_1}
\end{figure}

The first octet shows that both the source address \texttt{35.186.224.47} and
the destination address \texttt{10.0.04} belong to the A class.

Given that both addresses belong to the same class, and the last 3 octets are
different, we can infer a subnet mask of \texttt{255.0.0.0}.

The destination address might be a private one, as it follows the
\texttt{10.0.0.0/8} default pattern.

Finally, the TTL value of 57 might indicate that the device sending the packet
has a Linux OS.

\paragraph*{Pending evidence}
\begin{itemize}
    \item Screenshot of the IPv6 DNS request/response (Figure~\ref{fig:exp2_ipv6_flow}).
\end{itemize}
