\subsection{First Experience}

\subsubsection{Capturing packets from an execution of traceroute}
Dado que el docente ya había preparado un rastro listo para analizar, seguimos
exactamente los pasos del PDF usando el archivo \texttt{ip-ethereal-trace-1}
que está en la carpeta \texttt{wireshark-traces}. Allí se aprecia cómo cada
proba de \texttt{traceroute} (desde \texttt{192.168.1.102}) sale con un TTL que
crece de uno en uno, y cómo los routers intermedios responden con ICMP ``time
exceeded'' conforme el TTL llega a cero. Las primeras tres sondas mueren en el
gateway, las siguientes tres alcanzan el segundo salto, y así sucesivamente
hasta que un lote logra llegar al destino y se ve claramente el cambio de IP en
la columna ``Source'' de Wireshark. La figura~\ref{fig:exp1_traceroute_cli}
resume exactamente esa salida de \texttt{traceroute}.

Lo interesante es que, salvo el TTL y el checksum, todo lo demás permanece
constante en cada datagrama: mismas direcciones, mismo tamaño y mismo
identificador ICMP. Eso hace que el patrón ``tres requests verdes seguidos de
un reply rojo'' sea muy evidente en la vista de paquetes y que el comportamiento
descrito en el enunciado quede totalmente ligado a lo que muestra la captura.

\begin{figure}[htbp]
    \centering
    \fbox{\parbox{0.9\linewidth}{\centering Salida del traceroute provisto en el laboratorio (saltos hacia \texttt{gaia.cs.umass.edu}).}}
    \caption{Traceroute del archivo \texttt{ip-ethereal-trace-1}}\label{fig:exp1_traceroute_cli}
\end{figure}

\subsubsection{Basic IPv4}
Zooming into the very first probe in \texttt{ip-ethereal-trace-1} shows how
little metadata traceroute actually needs. The IPv4 header is just the base
20-byte structure (IHL~=~5), the protocol field is set to 1 because this
Windows box uses ICMP echo requests, and the total length hovers around 98~bytes
(20 bytes of IPv4, 8 bytes of ICMP, and a short payload that the app uses as a
sequence tag). The source-destination pair is
\texttt{192.168.1.102$\rightarrow$128.59.23.100}, which matches the MAC/ARP
exchange at the beginning of the file. Only a handful of fields vary between
successive probes: the TTL (1, then 2, then 3), the checksum (because TTL is
part of it) and the ICMP identifier/sequence tuple that traceroute uses to map
each reply back to the correct line in the CLI output.

Everything else stays constant. The DF flag never flips because the tool leaves
fragmentation to routers, the version and header length obviously remain fixed,
and the destination address never changes. That stability makes it easy to
visually isolate the TTL-induced losses in the packet list—filtering for ICMP
and sorting by time gives a ladder-like pattern where every three requests are
followed by a red ``TTL~exceeded'' row. Figure~\ref{fig:exp1_ttl_icmp} is a
placeholder for that filtered view so we can add the screenshot right before
turning in the PDF.

\begin{figure}[htbp]
    \centering
    \fbox{\parbox{0.9\linewidth}{\centering Vista filtrada en Wireshark con los ICMP ``Time Exceeded'' del traceroute.}}
    \caption{Paquetes ICMP TTL Exceeded en la traza}\label{fig:exp1_ttl_icmp}
\end{figure}

\subsubsection{Fragmentation}

This section of the experience was done on a Windows machine, and so we had to
work with the downloaded trace file.

First, we looked for the first IP datagram containing the first part of the
segment sent to the server. After finding it (it wasn't the first packet), we
looked for any field that might indicate that it was fragmented.

\begin{figure}[htbp]
    \centering
    \includegraphics[width=1\linewidth]{img/1/3_1.png}
    \caption{First fragment}\label{fig:exp1_3_1}
\end{figure}

We can see that the flag \textbf{More Fragments} is set, and thus we can
conclude that the packet is indeed fragmented.

We then wanted to know if this packet was the first fragment, or a latter one.
Seeing that the \textbf{fragment offset} value is zero, we can conclude that
this is the first fragment.

The total number of bytes in this datagram (payload plus header) is 1480 plus
20, i.e. 1500, as shown in the previous image.

Moving on to the second fragment, we can see that the fragment offset is no
longer zero. This implies that this fragment is not the first one (which is
true).

\begin{figure}[htbp]
    \centering
    \includegraphics[width=1\linewidth]{img/1/3_2.png}
    \caption{Second fragment}\label{fig:exp1_3_2}
\end{figure}

If we compare the first and second fragments, we can identify the following
differences:

\begin{itemize}
    \item Fragment offset: 0 for first, 1480 for second
    \item Header checksum: different values as the fragment offset is different
\end{itemize}

Finally, we observed the third's fragment information. We immediately notice
that the flag values are different. In this case, the flag \textbf{more
    fragments} is unset, which implies that this is indeed the last fragment.

\begin{figure}[htbp]
    \centering
    \includegraphics[width=1\linewidth]{img/1/3_3.png}
    \caption{Third fragment}\label{fig:exp1_3_3}
\end{figure}

\paragraph*{Pending evidence}
\begin{itemize}
    \item Screenshot of the traceroute executions (Figure~\ref{fig:exp1_traceroute_cli}).
    \item Wireshark capture filtered by \texttt{icmp} (Figure~\ref{fig:exp1_ttl_icmp}) highlighting the returned hops.
\end{itemize}
