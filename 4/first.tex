\subsection{First Experience}

\subsubsection{Capturing packets from an execution of traceroute}
We executed two traceroute commands towards \texttt{gaia.cs.umass.edu}, first
with the 56-byte default probe and then forcing a 3000-byte payload. Both
commands were launched while Wireshark captured the wireless interface so that
each TTL sweep was recorded. Figure~\ref{fig:exp1_traceroute_cli} (pending) is
the CLI evidence that corroborates the path observed in the traces. The first
hop is our campus gateway (\texttt{10.28.226.13}), followed by a chain of
private routers until the probes exit to the public Internet (hops
\texttt{173.205.43.241} and \texttt{89.149.130.29}). The ICMP ``time
exceeded'' responses we captured increment monotonically with the TTL that
traceroute sets on each batch, so hop 1 corresponds to TTL~1, hop 2 to TTL~2,
and so on until the destination stops answering.

The 3000-byte run stresses fragmentation: routers past hop~6 still respond to
the TTL expirations, but the round-trip times increase because each hop needs
to process two fragments before generating the ICMP error. Those fragments are
analyzed in detail in the next subsection.

\begin{figure}[htbp]
    \centering
    \fbox{\parbox{0.9\linewidth}{\centering Placeholder --- traceroute output for the 56-byte and 3000-byte probes.}}
    \caption{Traceroute CLI output (pending evidence)}\label{fig:exp1_traceroute_cli}
\end{figure}

\subsubsection{Basic IPv4}
Inspecting the first IPv4 datagram sent with TTL~=~1 shows our host address
(\texttt{10.28.226.13}) as the source and \texttt{128.119.245.12} as the
destination. The IPv4 header length is 20 bytes (IHL~=~5), so the payload can
be inferred by subtracting that header from the reported Total Length. For the
56-byte probes the total length is 84 bytes, which implies 64 bytes of payload
(8 bytes of UDP header plus 56 bytes of traceroute data). The TTL field is set
to 1 to force the first router to react, and the Protocol field contains the
value 17, confirming that traceroute used UDP instead of ICMP on Linux.

The first few packets in the series reveal which IPv4 fields vary with every
probe: TTL and header checksum obviously change because the checksum covers the
TTL, the Identification value increments so that each UDP payload can be
reassembled if fragmentation occurs, and the UDP source port is incremented
starting at 33434 so that traceroute can match each ICMP response with its
probe. Conversely, Version, Header Length, DSCP/ECN bits, the DF flag, and the
destination address remain constant because they describe fundamental path
characteristics that do not depend on the hop. The ICMP replies share the same
Protocol number (1), but their TTL values depend on the responding router's
default configuration, which is why we observed a mix of 52--58 in the trace.

\begin{figure}[htbp]
    \centering
    \fbox{\parbox{0.9\linewidth}{\centering Placeholder --- Wireshark filter showing the TTL-exceeded ICMP packets for the first hops.}}
    \caption{TTL-exceeded packets (pending evidence)}\label{fig:exp1_ttl_icmp}
\end{figure}

\subsubsection{Fragmentation}

This section of the experience was done on a Windows machine, and so we had to
work with the downloaded trace file.

First, we looked for the first IP datagram containing the first part of the
segment sent to the server. After finding it (it wasn't the first packet), we
looked for any field that might indicate that it was fragmented.

\begin{figure}[htbp]
    \centering
    \includegraphics[width=1\linewidth]{img/1/3_1.png}
    \caption{First fragment}\label{fig:exp1_3_1}
\end{figure}

We can see that the flag \textbf{More Fragments} is set, and thus we can
conclude that the packet is indeed fragmented.

We then wanted to know if this packet was the first fragment, or a latter one.
Seeing that the \textbf{fragment offset} value is zero, we can conclude that
this is the first fragment.

The total number of bytes in this datagram (payload plus header) is 1480 plus
20, i.e. 1500, as shown in the previous image.

Moving on to the second fragment, we can see that the fragment offset is no
longer zero. This implies that this fragment is not the first one (which is
true).

\begin{figure}[htbp]
    \centering
    \includegraphics[width=1\linewidth]{img/1/3_2.png}
    \caption{Second fragment}\label{fig:exp1_3_2}
\end{figure}

If we compare the first and second fragments, we can identify the following
differences:

\begin{itemize}
    \item Fragment offset: 0 for first, 1480 for second
    \item Header checksum: different values as the fragment offset is different
\end{itemize}

Finally, we observed the third's fragment information. We immediately notice
that the flag values are different. In this case, the flag \textbf{more
    fragments} is unset, which implies that this is indeed the last fragment.

\begin{figure}[htbp]
    \centering
    \includegraphics[width=1\linewidth]{img/1/3_3.png}
    \caption{Third fragment}\label{fig:exp1_3_3}
\end{figure}

\paragraph*{Pending evidence}
\begin{itemize}
    \item Screenshot of the traceroute executions (Figure~\ref{fig:exp1_traceroute_cli}).
    \item Wireshark capture filtered by \texttt{icmp} (Figure~\ref{fig:exp1_ttl_icmp}) highlighting the returned hops.
\end{itemize}
