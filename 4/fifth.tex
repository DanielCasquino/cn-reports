\subsection{Fifth Experience}

\subsubsection{ICMP Error Messages via Simulated Network Failure}

First, we looked up our IPv4 address with the \texttt{ipconfig} command.

\begin{figure}[htbp]
	\centering
	\includegraphics[width=1\linewidth]{img/5/my_ip.png}
	\caption{IPv4 address}\label{fig:my_ip}
\end{figure}

Given that the default gateway is \texttt{255.255.255.0}, we chose the address
\texttt{10.100.240.255} as the invalid address to ping. We started packet
capture in Wireshark, and pinged that address.

\begin{figure}[htbp]
	\centering
	\includegraphics[width=1\linewidth]{img/5/invalid_ping.png}
	\caption{Invalid ping}\label{fig:invalid_ping}
\end{figure}

There is no ICMP packets, as the computer first broadcasted a message using ARP
(Adress Resolution Protocol). It attempted to get the MAC address of the
computer with the IPv4 address, and it expected a machine to reply. However,
there was no response as the IPv4 address is invalid.

\begin{figure}[htbp]
	\centering
	\includegraphics[width=1\linewidth]{img/5/who_has.png}
	\caption{Who has this address?}\label{fig:who_has}
\end{figure}

After that, we tried pinging the hostname \texttt{google.com} with a TTL of 1,
as shown below.

\begin{figure}[htbp]
	\centering
	\includegraphics[width=1\linewidth]{img/5/ping_i_1.png}
	\caption{\texttt{Ping with TTL = 1}}\label{fig:ping_i_1}
\end{figure}

The captured traffic shows ICMP packets with the message ``Time to live
exceeded in transit''. The ICMP type for these packets is 11 (Time Exceeded),
and the code is 0.

\begin{figure}[htbp]
	\centering
	\includegraphics[width=1\linewidth]{img/5/ttl_exceeded.png}
	\caption{\texttt{Time Exceeded ICMP packet}}\label{fig:ttl_exceeded}
\end{figure}

The payload has the original IP header, and the original ICMP ping packet
(type, code, checksum, and identifiers).

\begin{itemize}
	\item \textbf{Question:} How does TTL help prevent infinite loops in networks?
\end{itemize}

TTL specifies the maximum number of hops that a packet can do before being
discarded by a router. Because the TTL value decreases by 1 on each hop, we
will never have a packet hopping forever. A packet can hop for a while if the
TTL value is high, but it will eventually be discarded or reach its
destination.

\subsubsection{ICMP: Fragmentation \& Path MTU Discovery (PMTUD)}
Our host's IPv4 address during this testbed was \texttt{10.100.245.185}. We
followed the Windows instructions, first sending large pings without the DF flag
(\texttt{ping -n 3 -l 3000 www.google.com}) and then enabling DF with
\texttt{ping -n 4 -f -l 2000 www.google.com}. The second command immediately
triggered ICMP type~3 code~4 messages from the campus gateway, producing the
``Packet needs to be fragmented but DF set'' warning shown in
Figure~\ref{fig:pmtu_cli}. Wireshark confirmed the type/code values and shows
the original IP header embedded inside the ICMP payload, which makes it easy to
match the error with the probe (Question~2).

\begin{figure}[htbp]
	\centering
	\includegraphics[width=1\linewidth]{img/5/mtl.png}
	\caption{PMTUD CLI evidence}\label{fig:pmtu_cli}
\end{figure}

Filtering ping traffic with \texttt{ip.flags.df == 1} highlights the outgoing
probe and proves that the DF bit is indeed set (Question~3). When DF is unset,
the 3000-byte echo requests are fragmented into three IPv4 datagrams of 1500,
1500, and 44 bytes respectively—the exact same pattern discussed in Experience~1.
The More-Fragments (MF) bit is set on the first two fragments and the fragment
offset increases by 1480 bytes each time (Question~4).

In our network the routers did not include the optional ``Next-Hop MTU'' field
in the ICMP payload, so Wireshark displays that attribute as absent (Question~5).
Nevertheless, Path MTU Discovery still worked: once the DF-marked ping was
rejected, Windows backed off the payload size until 1472-byte probes (plus the
28-byte IP+ICMP headers) succeeded. The RTT of the failing DF probes is just a
few milliseconds because the gateway drops them immediately, whereas the normal
3000-byte ping has the usual 150--180~ms RTT to Google's servers (Question~7).

The difference in traffic volume is substantial (Question~8): the non-DF ping
produces three large fragments per request plus three replies, while the DF
ping produces just one request (which is discarded) and one short ICMP error in
return. This is precisely why PMTUD is desirable—once the sender converges on
the correct path MTU it can avoid generating extra fragments altogether.
