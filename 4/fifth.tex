\subsection{Fifth Experience}

\subsubsection{ICMP Error Messages via Simulated Network Failure}

First, we looked up our IPv4 address with the \texttt{ipconfig} command.

\begin{figure}[htbp]
	\centering
	\includegraphics[width=1\linewidth]{img/5/my_ip.png}
	\caption{IPv4 address}\label{fig:my_ip}
\end{figure}

Given that the default gateway is \texttt{255.255.255.0}, we chose the address
\texttt{10.100.240.255} as the invalid address to ping. We started packet
capture in Wireshark, and pinged that address.

\begin{figure}[htbp]
	\centering
	\includegraphics[width=1\linewidth]{img/5/invalid_ping.png}
	\caption{Invalid ping}\label{fig:invalid_ping}
\end{figure}

There is no ICMP packets, as the computer first broadcasted a message using ARP
(Adress Resolution Protocol). It attempted to get the MAC address of the
computer with the IPv4 address, and it expected a machine to reply. However,
there was no response as the IPv4 address is invalid.

\begin{figure}[htbp]
	\centering
	\includegraphics[width=1\linewidth]{img/5/who_has.png}
	\caption{Who has this address?}\label{fig:who_has}
\end{figure}

After that, we tried pinging the hostname \texttt{google.com} with a TTL of 1,
as shown below.

\begin{figure}[htbp]
	\centering
	\includegraphics[width=1\linewidth]{img/5/ping_i_1.png}
	\caption{\texttt{Ping with TTL = 1}}\label{fig:ping_i_1}
\end{figure}

The captured traffic shows ICMP packets with the message ``Time to live
exceeded in transit''. The ICMP type for these packets is 11 (Time Exceeded),
and the code is 0.

\begin{figure}[htbp]
	\centering
	\includegraphics[width=1\linewidth]{img/5/ttl_exceeded.png}
	\caption{\texttt{Time Exceeded ICMP packet}}\label{fig:ttl_exceeded}
\end{figure}

The payload has the original IP header, and the original ICMP ping packet
(type, code, checksum, and identifiers).

\begin{itemize}
	\item \textbf{Question:} How does TTL help prevent infinite loops in networks?
\end{itemize}

TTL specifies the maximum number of hops that a packet can do before being
discarded by a router. Because the TTL value decreases by 1 on each hop, we
will never have a packet hopping forever. A packet can hop for a while if the
TTL value is high, but it will eventually be discarded or reach its
destination.

\subsubsection{ICMP: Fragmentation \& Path MTU Discovery (PMTUD)}
The IP address of my host is 10.100.245.185. We used multiple destination addresses, one of them, for example, was 0.0.0.1. There's no packet reporting fragmentation needed. In our capture, there where no ICMP packets width type 3 and code 4 (fragmentation neededed), probably because the . For an example packet, we can see the number of fragments depends on the MTU and the size of the packet. In this case
