\subsection{Third Experience}

\subsubsection{NAT Lab with Wireshark}

After visiting \url{https://whatsmyipaddress.com}, we obtained the following
addresses.

\begin{figure}[htbp]
	\centering
	\includegraphics[width=1\linewidth]{img/3/ip_utec.png}
	\caption{Public IP on uni network}\label{fig:ip_uni}
\end{figure}

\begin{figure}[htbp]
	\centering
	\includegraphics[width=1\linewidth]{img/3/ip_hotspot.png}
	\caption{Public IP on hotspot}\label{fig:ip_hotspot}
\end{figure}

Both laptops show the same public address despite being on the same network.
This is possible thanks to NAT. That is, all devices under the UTEC network
share the same \textbf{public} IP address, and communication is instead
identified via port usage (thanks to the router). Devices do have different
\textbf{private} address.

If an outside device attempted to stablish communication with a private IP, it
would fail, as it would try to find a device inside its own private network.

Given that all devices share the same public IP, the router uses ports to
differentiate between them. For instance, two devices might use port 5000 and
5001, even when sharing the same public IP. The user header is not ``port'',
but rather ``Identification''.

The port numbers used for the \texttt{8.8.8.8} ping were \texttt{5235}.

We can determine the protocol used (IPv4 or IPv6) by checking either the IP
format, or the packet type (ICMP or ICMPv6).

The captured packets involve the use of the ICMP protocol, as it is being used
to send pings to other devices.

NAT is part of the network layer, as it deals with defining where data is sent.

\begin{figure}[htbp]
	\centering
	\includegraphics[width=1\linewidth]{img/3/http_comp.png}
	\caption{HTTP}\label{fig:http_comp}
\end{figure}

We can see that http uses a port, but NAT uses another one to map the private
ip to something that is publicly addressable. It maps an (address, port)
private tuple to a public tuple.

% My private ip is 10.100.255.35


\subsubsection{Real NAT Application Analysis}

We loaded the \texttt{NAT\_home\_side} and \texttt{NAT\_ISP\_side} traces
provided with the lab and focused on the HTTP GET issued at $t=7.109\,\mathrm{s}$
in the home capture. Filtering with \texttt{http \&\& ip.addr==64.233.169.104}
keeps only the traffic to Google's frontend. Figure~\ref{fig:nat_home_filter}
shows the context from the inside of the LAN: the client address is
\texttt{192.168.1.100} and its TCP port is \texttt{4335}.

\begin{figure}[htbp]
	\centering
	\includegraphics[width=1\linewidth]{img/3/dg-home.png}
	\caption{Home-side filter focusing on the Google flow}\label{fig:nat_home_filter}
\end{figure}

For the GET at time $7.109\,\mathrm{s}$ the IPv4 header confirms a source of
\texttt{192.168.1.100}, a destination of \texttt{64.233.169.104}, TTL~64 and protocol
17 (UDP) for the preceding DNS exchanges, followed by protocol 6 (TCP) for the
HTTP session. The TCP tuple is \texttt{4335$\rightarrow$80}. The corresponding
200~OK at $7.158\,\mathrm{s}$ reuses the same ports and reverses the IPs.
Figures~\ref{fig:nat_home_get} and~\ref{fig:nat_home_ok} summarize both
packets. Those screenshots answer questions 2--4 from the handout.

\begin{figure}[htbp]
	\centering
	\includegraphics[width=1\linewidth]{img/3/get.png}
	\caption{HTTP GET observed inside the LAN}\label{fig:nat_home_get}
\end{figure}

\begin{figure}[htbp]
	\centering
	\includegraphics[width=1\linewidth]{img/3/ok.png}
	\caption{First 200~OK inside the LAN}\label{fig:nat_home_ok}
\end{figure}

Before those HTTP messages the usual three-way handshake takes place:
frame~53 contains the SYN at $t=7.075\,\mathrm{s}$, frame~54 is the server's
SYN/ACK at $t=7.109\,\mathrm{s}$ (arriving from \texttt{64.233.169.104} to
\texttt{192.168.1.100}), and frame~55 is the ACK that completes the
handshake~(Figure~\ref{fig:nat_home_syn}). This answers questions 5--7.

\begin{figure}[htbp]
	\centering
	\includegraphics[width=1\linewidth]{img/3/syn.png}
	\caption{SYN/SYN-ACK/ACK sequence on the home side}\label{fig:nat_home_syn}
\end{figure}

Switching to the ISP-facing capture reveals the effect of NAT. The same GET now
uses the public address \texttt{71.192.34.104} but retains TCP port~4335. It
appears at $t=6.069\,\mathrm{s}$ because the ISP trace has its own reference
clock. Figure~\ref{fig:nat_isp_get} shows that Version, Header Length, Flags,
and TCP options are identical to the home-side datagram, but the source address
and checksum changed (the checksum must be recomputed whenever a header field
is rewritten). The 200~OK at $t=6.118\,\mathrm{s}$ mirrors those fields as well
(Figure~\ref{fig:nat_isp_ok}).

\begin{figure}[htbp]
	\centering
	\includegraphics[width=1\linewidth]{img/3/get-isp.png}
	\caption{Same GET after NAT translation}\label{fig:nat_isp_get}
\end{figure}

\begin{figure}[htbp]
	\centering
	\includegraphics[width=1\linewidth]{img/3/ok-isp.png}
	\caption{200~OK at the ISP tap}\label{fig:nat_isp_ok}
\end{figure}

The SYN in this trace is frame~82 ($t=6.035\,\mathrm{s}$), followed by the
server's SYN/ACK (frame~83) and the translated ACK (frame~84). Ports remain
unchanged by this particular NAT because it only needs to translate a single
internal flow, but large deployments often remap the source port as well to
avoid collisions. Figure~\ref{fig:nat_isp_syn} and Figure~\ref{fig:nat_isp_dg}
capture both the timing and the translated addresses.

\begin{figure}[htbp]
	\centering
	\includegraphics[width=1\linewidth]{img/3/ack.png}
	\caption{SYN/SYN-ACK/ACK on the ISP side}\label{fig:nat_isp_syn}
\end{figure}

\begin{figure}[htbp]
	\centering
	\includegraphics[width=1\linewidth]{img/3/dg-isp.png}
	\caption{Packets selected at the ISP tap}\label{fig:nat_isp_dg}
\end{figure}

Comparing questions 7 and 12 therefore yields the expected differences: IP
addresses and header checksums change, but the TCP five-tuple, flags, and
payloads are untouched. Version, header length, and DF/MF flags are identical
across both captures, so the router performs a minimal translation without
disturbing the transport flow.
