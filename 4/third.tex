\subsection{Third Experience}

\subsubsection{NAT Lab with Wireshark}

After visiting \url{https://whatsmyipaddress.com}, we obtained the following
addresses.

\begin{figure}[htbp]
	\centering
	\includegraphics[width=1\linewidth]{img/3/ip_utec.png}
	\caption{Public IP on uni network}\label{fig:ip_uni}
\end{figure}

\begin{figure}[htbp]
	\centering
	\includegraphics[width=1\linewidth]{img/3/ip_hotspot.png}
	\caption{Public IP on hotspot}\label{fig:ip_hotspot}
\end{figure}

Both laptops show the same public address despite being on the same network.
This is possible thanks to NAT. That is, all devices under the UTEC network
share the same \textbf{public} IP address, and communication is instead
identified via port usage (thanks to the router). Devices do have different
\textbf{private} address.

If an outside device attempted to stablish communication with a private IP, it
would fail, as it would try to find a device inside its own private network.

Given that all devices share the same public IP, the router uses ports to
differentiate between them. For instance, two devices might use port 5000 and
5001, even when sharing the same public IP. The user header is not ``port'',
but rather ``Identification''.

The port numbers used for the \texttt{8.8.8.8} ping were \texttt{5235}.

We can determine the protocol used (IPv4 or IPv6) by checking either the IP
format, or the packet type (ICMP or ICMPv6).

The captured packets involve the use of the ICMP protocol, as it is being used
to send pings to other devices.

NAT is part of the network layer, as it deals with defining where data is sent.

\begin{figure}[htbp]
	\centering
	\includegraphics[width=1\linewidth]{img/3/http_comp.png}
	\caption{HTTP}\label{fig:http_comp}
\end{figure}

We can see that http uses a port, but NAT uses another one to map the private
ip to something that is publicly addressable. It maps an (address, port)
private tuple to a public tuple.

% My private ip is 10.100.255.35


\subsubsection{Real NAT Application Analysis}

The client ip address is 192.168.1.100

\begin{itemize}
	\item 10.15.32.200/8: class A, NA 10.0.0.0, BA 10.255.255.255
	\item 172.20.45.7/12: class B, NA 172.16.0.0, BA 172.16.255.255
	\item 192.168.10.25/16: class C, NA 192.168.0.0, BA 192.168.255.255
\end{itemize}

After choosing the first TCP packet, we expanded and examined the IPv4 section.

\begin{figure}[htbp]
	\centering
	\includegraphics[width=1\linewidth]{img/2/6_1.png}
	\caption{TCP packet's IPv4 section}\label{fig:exp2_6_1}
\end{figure}

The first octet shows that both the source address \texttt{35.186.224.47} and
the destination address \texttt{10.0.04} belong to the A class.

Given that both addresses belong to the same class, and the last 3 octets are
different, we can infer a subnet mask of \texttt{255.0.0.0}.

The destination address might be a private one, as it follows the
\texttt{10.0.0.0/8} default pattern.

Finally, the TTL value of 57 might indicate that the device sending the packet
has a Linux OS.
