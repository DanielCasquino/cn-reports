% The very first letter is a 2 line initial drop letter followed
% by the rest of the first word in caps.
% 
% form to use if the first word consists of a single letter:
% \IEEEPARstart{A}{demo} file is ....
% 
% form to use if you need the single drop letter followed by
% normal text (unknown if ever used by the IEEE):
% \IEEEPARstart{A}{}demo file is ....
% 
% Some journals put the first two words in caps:
% \IEEEPARstart{T}{his demo} file is ....
% 
% Here we have the typical use of a "T" for an initial drop letter
% and "HIS" in caps to complete the first word.
\IEEEPARstart{T}{he} network layer is responsible for stitching together
heterogeneous links, naming hosts with logical addresses, and ensuring that
datagrams advance hop by hop until they reach the correct transport endpoint.
Unlike the link layer, where framing details change every few meters of cable,
network-layer protocols must remain portable and scalable despite mobility,
address exhaustion, and middleboxes~\cite{tanenbaum:networks}. Every field in
an IPv4 or IPv6 header is therefore a lever that either preserves reachability
(time-to-live, checksums, flow labels), or equips upper layers with the context
they need to recover from loss and reordering.

This fourth lab block focused entirely on those levers. We executed controlled
traceroutes and MTU sweeps to observe fragmentation, captured DNS AAAA
transactions to study IPv6 semantics, instrumented NAT behavior on both sides
of a translation, injected ICMP errors, and forced DHCP to renegotiate leases.
The following sections consolidate those experiments into a single narrative so
that the choices we made during each checkpoint remain traceable.
% You must have at least 2 lines in the paragraph with the drop letter
% (should never be an issue)

%\hfill mds

%\hfill August 26, 2015

\subsection{Theoretical Framework}
% needed in second column of first page if using \IEEEpubid
%\IEEEpubidadjcol
\subsubsection{IPv4 Datagram Structure}
IPv4 datagrams carry a fixed 20-byte header followed by a variable payload. Key
fields include the Identification value (used to associate fragments), the Flags
field (with bits for ``Don't Fragment'' and ``More Fragments''), and the
Time-To-Live (TTL), which routers decrement to avoid loops. The Protocol field
demultiplexes into transport layers such as UDP, TCP or ICMP, and the header
checksum protects only the IPv4 header, not the payload~\cite{rfc791}.

\subsubsection{Fragmentation and Path MTU}
When a datagram larger than the outgoing link's Maximum Transmission Unit
arrives, routers must either fragment it or drop it if DF is set. Fragmentation
reuses the Identification, fragment offset (8-byte granularity) and MF flag to
allow the final host to reassemble the payload. Path MTU Discovery (PMTUD)
reduces reliance on in-network fragmentation by probing the largest DF-marked
packet that can traverse the path and adapting segment sizes accordingly,
typically at the transport layer~\cite{rfc791,rfc1191}.

\subsubsection{IPv6 Addressing and Flow Labels}
IPv6 expands addresses to 128 bits, removes header checksum and fragmentation
from routers, and introduces the Flow Label so that intermediate devices can
differentiate long-lived flows without peeking at transport headers. Extension
headers carry optional functionality (routing, fragmentation, security) while
neighbor discovery replaces ARP~\cite{rfc8200}. IPv6 also relies on stateless
address autoconfiguration in addition to DHCPv6 when appropriate.

\subsubsection{Network Address Translation}
NAT devices map private tuples (address, port) to public tuples so that multiple
hosts can share scarce public IPv4 addresses. Traditional ``NAPT'' rewrites
both addresses and transport ports, keeps per-flow state to demultiplex
responses, and times out idle mappings. While NAT simplifies address reuse, it
complicates inbound connectivity and any protocol that embeds literal IP
addresses inside payloads~\cite{rfc3022}.

\subsubsection{Internet Control Message Protocol}
ICMP piggybacks on IP to report errors and diagnostics. Type/Code pairs
distinguish echo requests/replies (ping), TTL exceeded (traceroute),
destination unreachable (including ``Fragmentation needed and DF set''), and
other control signals. Because ICMP messages include the failing packet's
header and the first eight bytes of payload, they provide enough context for
the originator to match the error with the triggering transmission~\cite{rfc792}.

\subsubsection{Dynamic Host Configuration Protocol}
DHCP automates IPv4 configuration by leasing addresses and network parameters
through a four-way Discover-Offer-Request-Ack handshake. Clients can request
specific options (router, DNS, lease time) through the parameter-request list,
and servers track leases so they can reclaim or renew addresses when hosts
leave the network~\cite{rfc2131}. Even though DHCP is defined at the
application layer, its sole purpose is to provision network-layer state.

% An example of a floating figure using the graphicx package.
% Note that \label must occur AFTER (or within) \caption.
% For figures, \caption should occur after the \includegraphics.
% Note that IEEEtran v1.7 and later has special internal code that
% is designed to preserve the operation of \label within \caption
% even when the captionsoff option is in effect. However, because
% of issues like this, it may be the safest practice to put all your
% \label just after \caption rather than within \caption{}.
%
% Reminder: the "draftcls" or "draftclsnofoot", not "draft", class
% option should be used if it is desired that the figures are to be
% displayed while in draft mode.
%
%\begin{figure}[!t]
%\centering
%\includegraphics[width=2.5in]{myfigure}
% where an .eps filename suffix will be assumed under latex, 
% and a .pdf suffix will be assumed for pdflatex; or what has been declared
% via \DeclareGraphicsExtensions.
%\caption{Simulation results for the network.}
%\label{fig_sim}
%\end{figure}

% Note that the IEEE typically puts floats only at the top, even when this
% results in a large percentage of a column being occupied by floats.

% An example of a double column floating figure using two subfigures.
% (The subfig.sty package must be loaded for this to work.)
% The subfigure \label commands are set within each subfloat command,
% and the \label for the overall figure must come after \caption.
% \hfil is used as a separator to get equal spacing.
% Watch out that the combined width of all the subfigures on a 
% line do not exceed the text width or a line break will occur.
%
%\begin{figure*}[!t]
%\centering
%\subfloat[Case I]{\includegraphics[width=2.5in]{box}%
%\label{fig_first_case}}
%\hfil
%\subfloat[Case II]{\includegraphics[width=2.5in]{box}%
%\label{fig_second_case}}
%\caption{Simulation results for the network.}
%\label{fig_sim}
%\end{figure*}
%
% Note that often IEEE papers with subfigures do not employ subfigure
% captions (using the optional argument to \subfloat[]), but instead will
% reference/describe all of them (a), (b), etc., within the main caption.
% Be aware that for subfig.sty to generate the (a), (b), etc., subfigure
% labels, the optional argument to \subfloat must be present. If a
% subcaption is not desired, just leave its contents blank,
% e.g., \subfloat[].

% An example of a floating table. Note that, for IEEE style tables, the
% \caption command should come BEFORE the table and, given that table
% captions serve much like titles, are usually capitalized except for words
% such as a, an, and, as, at, but, by, for, in, nor, of, on, or, the, to
% and up, which are usually not capitalized unless they are the first or
% last word of the caption. Table text will default to \footnotesize as
% the IEEE normally uses this smaller font for tables.
% The \label must come after \caption as always.
%
%\begin{table}[!t]
%% increase table row spacing, adjust to taste
%\renewcommand{\arraystretch}{1.3}
% if using array.sty, it might be a good idea to tweak the value of
% \extrarowheight as needed to properly center the text within the cells
%\caption{An Example of a Table}
%\label{table_example}
%\centering
%% Some packages, such as MDW tools, offer better commands for making tables
%% than the plain LaTeX2e tabular which is used here.
%\begin{tabular}{|c||c|}
%\hline
%One & Two\\
%\hline
%Three & Four\\
%\hline
%\end{tabular}
%\end{table}

% Note that the IEEE does not put floats in the very first column
% - or typically anywhere on the first page for that matter. Also,
% in-text middle ("here") positioning is typically not used, but it
% is allowed and encouraged for Computer Society conferences (but
% not Computer Society journals). Most IEEE journals/conferences use
% top floats exclusively. 
% Note that, LaTeX2e, unlike IEEE journals/conferences, places
% footnotes above bottom floats. This can be corrected via the
% \fnbelowfloat command of the stfloats package.

\subsection{State of the Art}
Large-scale measurements highlight how uneven IPv6 deployment still is: Czyz
et~al. correlate BGP announcements, DNS queries, and active probes to show that
while content providers are dual-stack, many residential ASes still lack native
IPv6 or fall back to tunnels, which has direct implications on latency and MTU
stability for AAAA-capable clients~\cite{czyz2014ipv6}. Middlebox transparency
also remains a research topic—tools such as Tracebox automatically reproduce
flows while recording every header rewrite along the path, revealing how NATs
and firewalls tamper with TTLs, checksums, and even TCP options in practice,
which in turn affects PMTUD and multipath transports~\cite{luckie2014tracebox}.

Finally, modern traffic increasingly traverses carrier-grade NATs, encrypted
DNS resolvers, and HTTP proxies. Razaghpanah et~al. instrumented thousands of
volunteer devices and showed that these middleboxes can introduce noticeable
latency, silently drop non-port-80 flows, and reduce entropy in the ephemeral
port space, which raises security concerns for scalable services~\cite{razaghpanah2018}.
The experiments in this report connect those findings with hands-on captures:
every trace we collected exhibits at least one of those middlebox behaviors.
