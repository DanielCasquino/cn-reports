\subsection{Research Questions}

\paragraph*{NAT and IPv6}
IPv6 reintroduces the end-to-end addressing model by allocating a virtually
unbounded space of global prefixes. Every interface can obtain multiple /64s,
and stateless address autoconfiguration already hides hardware identifiers via
temporary addresses. Therefore, the primary motivation for NAT---address
exhaustion---vanishes. Using NAT on IPv6 would also break extension headers and
defeat the simplified forwarding semantics that RFC~8200 promotes
~\cite{rfc8200,rfc3022}.

\paragraph*{Security posture in home networks}
Carrier-grade measurements show that middleboxes routinely mishandle flows
whenever they attempt to enforce policy by rewriting packets
~\cite{razaghpanah2018}. A home NAT provides a mild security benefit by
default-denying unsolicited inbound flows, but it cannot inspect encrypted
payloads or prevent lateral movement once a device is compromised. A hardened
home network therefore still needs host firewalls, timely patching, and, when
possible, network segmentation. NAT alone only obscures internal addressing; it
does not validate traffic.

\paragraph*{Is NAT a firewall?}
Traditional NAT rewrites source addresses and ports while maintaining a mapping
table. It is not stateful inspection: if a mapping exists, return traffic is
automatically forwarded, regardless of whether the payload is malicious or
unexpected. Likewise, once a port-forwarding rule is configured, NAT willingly
passes unsolicited traffic straight to the internal host. Firewalls, on the
other hand, enforce explicit policies on layer~3/4 headers (and often payloads)
regardless of address translation~\cite{rfc3022}. We therefore treat NAT as an
address conservation mechanism, not as a substitute for proper filtering.
