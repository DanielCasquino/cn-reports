\subsection{Report Research}

This subsection aims to answer two report questions, as well as provide
additional information otherwise excluded from the usual report content.

\subsubsection{Primary Purpose of DNS}

DNS (Domain Name System) is a hierarchy based system that uses domains to map
human-readable adresses (i.e.\@ www.google.com) to IP
addresses~\cite{tanenbaum:networks}. This is done to:

\begin{itemize}
    \item Make it easier for people to remember page addresses
    \item Enable servers to change addresses without affecting users
    \item Allow multiple hostnames to point to the same server or resource
    \item Provide a centralized way of managing domain names and avoiding hostname
          conflicts~\cite{tanenbaum:networks}
\end{itemize}

It's also this type of ``virtualization'' that allows the internet to support
such a large number of addresses, as the address pool of IPv4 (4.3 billion
addresses) was not enough. Even then, IPv6 had to be created to support an even
larger range of addresses.

To sum up, DNS works as a much needed layer of abstraction in the application
layer, allowing for decoupling of resources and their locations~\cite{rfc1034}.

\subsubsection{Domain Registration in Peru}

Domain Registration in Peru is often managed by third party companies, and the
registering process itself is quite simple for end users. Some of the main
hosting providers in Peru are:

\begin{itemize}
    \item GoDaddy
    \item DonWeb Peru
    \item Hosting Peru
    \item Punto.pe
\end{itemize}

The government also plays an important role in domain registration in specific
cases, such as for registering under the \textit{.gob.pe} domain, which is
reserved for government entities.In these cases, the requesting entity must
obtain authorization from the Government and Digital Transformation Secretariat
(SGTD) of the Presidency of the Council of Ministers
(PCM)~\cite{gobpe:dominio}. This involves a rather tedious bureaucratic
process, but this is done to avoid misuse of the domain.

Otherwise, registering under a domain such as \textit{.com} or \textit{.com.pe}
is quite simple. After choosing a provider and checking for domain
availability, it can be added to a cart and purchased with minimal
requirements~\cite{godaddy:peru}.