% The very first letter is a 2 line initial drop letter followed
% by the rest of the first word in caps.
% 
% form to use if the first word consists of a single letter:
% \IEEEPARstart{A}{demo} file is ....
% 
% form to use if you need the single drop letter followed by
% normal text (unknown if ever used by the IEEE):
% \IEEEPARstart{A}{}demo file is ....
% 
% Some journals put the first two words in caps:
% \IEEEPARstart{T}{his demo} file is ....
% 
% Here we have the typical use of a "T" for an initial drop letter
% and "HIS" in caps to complete the first word.
\IEEEPARstart{I}{P} addresses allow us to identify devices on a network,
and allow us to communicate with the precise device we want to reach.
However, there's only so many IP addresses that can be assigned and/or remembered, and as such,
a scalable solution is needed.

In these lab sessions, we experimented with the Domain Name System (DNS), which
functions as a system that maps strings of numbers to human-readable hostnames.
The goal of this report is to document and summarize our findings and
conclusions.
% You must have at least 2 lines in the paragraph with the drop letter
% (should never be an issue)

%\hfill mds

%\hfill August 26, 2015

\subsection{Theoretical Framework}
% needed in second column of first page if using \IEEEpubid
%\IEEEpubidadjcol
\subsubsection{Domain}
A domain can be defined as a group or subset of hostnames that are
hierarchically organized. There are several levels, starting from the root
domain, followed by the top-level domain, and finally authoritative and
subdomains~\cite{tanenbaum:networks}.

\subsubsection{Domain Name System (DNS)}
DNS is a protocol used for IP address resolution, translating human-readable
hostnames to IP addresses that can be used in a
network~\cite{tanenbaum:networks}.

\subsubsection{ipconfig}
\texttt{ipconfig} is a Windows command that displays
the current TCP/IP network configuration values, such as the device's IP address,
subnet mask, default gateway, and DNS servers~\cite{microsoft:ipconfig}.

\subsubsection{nslookup}
The \texttt{nslookup} command can be used to obtain domain name or IP address
mapping information. It can be used to find the IP address associated with a
hostname by querying DNS servers. It also supports setting the type of query
(NS, MX, A, CNAME)~\cite{microsoft:nslookup}.

% An example of a floating figure using the graphicx package.
% Note that \label must occur AFTER (or within) \caption.
% For figures, \caption should occur after the \includegraphics.
% Note that IEEEtran v1.7 and later has special internal code that
% is designed to preserve the operation of \label within \caption
% even when the captionsoff option is in effect. However, because
% of issues like this, it may be the safest practice to put all your
% \label just after \caption rather than within \caption{}.
%
% Reminder: the "draftcls" or "draftclsnofoot", not "draft", class
% option should be used if it is desired that the figures are to be
% displayed while in draft mode.
%
%\begin{figure}[!t]
%\centering
%\includegraphics[width=2.5in]{myfigure}
% where an .eps filename suffix will be assumed under latex, 
% and a .pdf suffix will be assumed for pdflatex; or what has been declared
% via \DeclareGraphicsExtensions.
%\caption{Simulation results for the network.}
%\label{fig_sim}
%\end{figure}

% Note that the IEEE typically puts floats only at the top, even when this
% results in a large percentage of a column being occupied by floats.

% An example of a double column floating figure using two subfigures.
% (The subfig.sty package must be loaded for this to work.)
% The subfigure \label commands are set within each subfloat command,
% and the \label for the overall figure must come after \caption.
% \hfil is used as a separator to get equal spacing.
% Watch out that the combined width of all the subfigures on a 
% line do not exceed the text width or a line break will occur.
%
%\begin{figure*}[!t]
%\centering
%\subfloat[Case I]{\includegraphics[width=2.5in]{box}%
%\label{fig_first_case}}
%\hfil
%\subfloat[Case II]{\includegraphics[width=2.5in]{box}%
%\label{fig_second_case}}
%\caption{Simulation results for the network.}
%\label{fig_sim}
%\end{figure*}
%
% Note that often IEEE papers with subfigures do not employ subfigure
% captions (using the optional argument to \subfloat[]), but instead will
% reference/describe all of them (a), (b), etc., within the main caption.
% Be aware that for subfig.sty to generate the (a), (b), etc., subfigure
% labels, the optional argument to \subfloat must be present. If a
% subcaption is not desired, just leave its contents blank,
% e.g., \subfloat[].

% An example of a floating table. Note that, for IEEE style tables, the
% \caption command should come BEFORE the table and, given that table
% captions serve much like titles, are usually capitalized except for words
% such as a, an, and, as, at, but, by, for, in, nor, of, on, or, the, to
% and up, which are usually not capitalized unless they are the first or
% last word of the caption. Table text will default to \footnotesize as
% the IEEE normally uses this smaller font for tables.
% The \label must come after \caption as always.
%
%\begin{table}[!t]
%% increase table row spacing, adjust to taste
%\renewcommand{\arraystretch}{1.3}
% if using array.sty, it might be a good idea to tweak the value of
% \extrarowheight as needed to properly center the text within the cells
%\caption{An Example of a Table}
%\label{table_example}
%\centering
%% Some packages, such as MDW tools, offer better commands for making tables
%% than the plain LaTeX2e tabular which is used here.
%\begin{tabular}{|c||c|}
%\hline
%One & Two\\
%\hline
%Three & Four\\
%\hline
%\end{tabular}
%\end{table}

% Note that the IEEE does not put floats in the very first column
% - or typically anywhere on the first page for that matter. Also,
% in-text middle ("here") positioning is typically not used, but it
% is allowed and encouraged for Computer Society conferences (but
% not Computer Society journals). Most IEEE journals/conferences use
% top floats exclusively. 
% Note that, LaTeX2e, unlike IEEE journals/conferences, places
% footnotes above bottom floats. This can be corrected via the
% \fnbelowfloat command of the stfloats package.

\subsection{State of the Art}
The Domain Name System (DNS) is a fundamental Internet service that
traditionally operates over the User Datagram Protocol (UDP). Classic DNS over
UDP offers no encryption or authentication, making queries and responses
vulnerable to eavesdropping and forgery arxiv.org researchgate.net. In the past
five years, extensive research and standardization efforts have focused on
improving DNS security, privacy, and performance. Notably, several secure DNS
protocols have been introduced: DNS Security Extensions (DNSSEC) for data
authenticity, DNS over TLS (DoT), DNS over HTTPS (DoH), and DNS over QUIC
(DoQ). These protocols encrypt DNS traffic (partially or fully) to protect
users from on-path attackers arxiv.org. Each comes with trade-offs in latency
and resource overhead arxiv.org , but they address the critical lack of
confidentiality and integrity in traditional DNS. Recent studies show that the
performance penalty of DNS encryption can be minimal. For example, DNS over
QUIC – a UDP-based encrypted transport – was found to load simple webpages up
to 10researchgate.net . On the other hand, researchers have noted that even
with encryption between client and resolver, the resolver can still observe
client identities and queries. To tackle this, oblivious DNS protocols (ODNS
and Oblivious DoH/DoQ) have been proposed, which use an intermediate proxy or
relay to hide the end-user’s IP address from the recursive resolver arxiv.org .
This approach prevents the DNS server from linking queries to specific clients,
further enhancing privacy at some additional complexity.% faster than DoH and to be only ~2% slower than unencrypted UDP DNS, making privacy-preserving DNS highly viable in practice

Beyond encryption, contemporary DNS research also explores reliability and new
applications of DNS. In Internet-of-Things (IoT) environments, DNS must operate
on resource-constrained devices and often lossy networks. A recent study by
Aydeger et al. (2025) evaluated DNSSEC, DoT, DoH, and DoQ on an edge IoT
resolver, comparing their performance under various conditions (with and
without caching) arxiv.org . The results highlight that secure DNS protocols
incur overhead (in processing and energy use) on IoT devices arxiv.org , but
they are necessary to protect IoT ecosystems from attacks like cache poisoning.
Indeed, millions of IoT devices have been found vulnerable due to outdated DNS
implementations (e.g. the Name:Wreck flaws) arxiv.org , underscoring the need
for robust DNS security in this domain. Another line of work has looked at
improving UDP itself to better support services like DNS. Researchers have
proposed enhanced UDP mechanisms with acknowledgments to increase reliability
for critical transactions. For instance, a 2024 experiment demonstrated that an
augmented UDP protocol significantly reduces packet loss compared to standard
UDP, thereby improving DNS delivery reliability in dynamic network topologies
(at the cost of slightly higher power consumption on devices) mdpi.com . This
indicates that even as DNS moves toward newer transports and security layers,
optimizations at the UDP layer can benefit scenarios where UDP remains in use.

Finally, the evolving DNS landscape has raised new security considerations.
Encrypting DNS traffic (via DoT/DoH/DoQ) mitigates privacy risks for users, but
it also complicates network defense, since traditional security tools can no
longer inspect DNS queries easily. Recent research provides insight into this
trade-off. Lyu et al. (2022) survey the DNS encryption literature and note that
malware has begun exploiting encrypted DNS channels for command-and-control and
data exfiltration, knowing that DNS queries are hidden from many monitoring
tools researchgate.net . To counter this, researchers are developing inference
techniques and out-of-band analyses to detect malicious patterns in encrypted
DNS traffic researchgate.net . In summary, the state-of-the-art DNS research
balances enhancements in security/privacy and performance: introducing
encryption and new protocols to protect DNS queries, refining UDP and resolver
strategies for reliability, and devising methods to manage the unintended side
effects of a more secure DNS ecosystem.

ejemplo de cita~\cite{rfc9114,rfc9000,rfc8446}.
