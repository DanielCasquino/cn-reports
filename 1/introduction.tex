% The very first letter is a 2 line initial drop letter followed
% by the rest of the first word in caps.
% 
% form to use if the first word consists of a single letter:
% \IEEEPARstart{A}{demo} file is ....
% 
% form to use if you need the single drop letter followed by
% normal text (unknown if ever used by the IEEE):
% \IEEEPARstart{A}{}demo file is ....
% 
% Some journals put the first two words in caps:
% \IEEEPARstart{T}{his demo} file is ....
% 
% Here we have the typical use of a "T" for an initial drop letter
% and "HIS" in caps to complete the first word.
\IEEEPARstart{H}{ttp} is one of the most widely used protocols for network communication, and
serves as the base for network data transmission. As such, understanding its inner workings and
applications is essential.

Through lab sessions in the Computer Networks course, we experimented with
Wireshark, capturing and analyzing the packets that our devices sent and
received. This report aims to summarize our experiences, as well as discuss our
results with the goal of further understanding HTTP.\@
% You must have at least 2 lines in the paragraph with the drop letter
% (should never be an issue)

%\hfill mds

%\hfill August 26, 2015

\subsection{Theoretical Framework}
% needed in second column of first page if using \IEEEpubid
%\IEEEpubidadjcol

\subsubsection{OSI Model}
The OSI (Open Systems Interconnection) model, developed by the ISO
(International Standards Organization), is a model that establishes standards
for network communication between systems. It consists of 7 layers:
Application, Presentation, Session, Transport, Network, Data Link, and
Physical~\cite{tanenbaum:networks}.

\subsubsection{IP Address}
An IP address is a unique identifier that represents a device on a network. An
IPv4 address is 32 bits long, and is written in dotted decimal notation; that
is, the address is divided into groups of 4 bytes, each one written in decimal
and divided by dots~\cite{tanenbaum:networks}.

\subsubsection{TCP}
The Transmission Control Protocol (TCP) is a transport protocol designed for
reliable process-to-process communication in a
network~\cite{tanenbaum:networks}. It enables byte transfer by packing them
into segments, and ensures that all data is received by assigning an
indentifier to each segment, thus allowing the sender to retransmit lost
segments and the receiver to reorder them or request their
retransmission.~\cite{rfc793}.

\subsubsection{UDP}
Subsubsection text here.

\subsubsection{HTTP}
HTTP (Hypertext Transfer Protocol) is an application level protocol that
operates over TCP and defines the format of requests and responses for messages
between servers and clients. It uses headers and methods (GET, POST, DELETE,
etc) to carry extra data for specific needs.~\cite{rfc9110}.

\subsubsection{Packet}
Subsubsection text here.

\subsubsection{Wireshark}
Wireshark is a FOSS for network packet analysis. Available in Linux, Windows,
and macOS, it allows users to capture and display live packets being sent and
received by any network device in their computers~\cite{wireshark:guide}.

% An example of a floating figure using the graphicx package.
% Note that \label must occur AFTER (or within) \caption.
% For figures, \caption should occur after the \includegraphics.
% Note that IEEEtran v1.7 and later has special internal code that
% is designed to preserve the operation of \label within \caption
% even when the captionsoff option is in effect. However, because
% of issues like this, it may be the safest practice to put all your
% \label just after \caption rather than within \caption{}.
%
% Reminder: the "draftcls" or "draftclsnofoot", not "draft", class
% option should be used if it is desired that the figures are to be
% displayed while in draft mode.
%
%\begin{figure}[!t]
%\centering
%\includegraphics[width=2.5in]{myfigure}
% where an .eps filename suffix will be assumed under latex, 
% and a .pdf suffix will be assumed for pdflatex; or what has been declared
% via \DeclareGraphicsExtensions.
%\caption{Simulation results for the network.}
%\label{fig_sim}
%\end{figure}

% Note that the IEEE typically puts floats only at the top, even when this
% results in a large percentage of a column being occupied by floats.

% An example of a double column floating figure using two subfigures.
% (The subfig.sty package must be loaded for this to work.)
% The subfigure \label commands are set within each subfloat command,
% and the \label for the overall figure must come after \caption.
% \hfil is used as a separator to get equal spacing.
% Watch out that the combined width of all the subfigures on a 
% line do not exceed the text width or a line break will occur.
%
%\begin{figure*}[!t]
%\centering
%\subfloat[Case I]{\includegraphics[width=2.5in]{box}%
%\label{fig_first_case}}
%\hfil
%\subfloat[Case II]{\includegraphics[width=2.5in]{box}%
%\label{fig_second_case}}
%\caption{Simulation results for the network.}
%\label{fig_sim}
%\end{figure*}
%
% Note that often IEEE papers with subfigures do not employ subfigure
% captions (using the optional argument to \subfloat[]), but instead will
% reference/describe all of them (a), (b), etc., within the main caption.
% Be aware that for subfig.sty to generate the (a), (b), etc., subfigure
% labels, the optional argument to \subfloat must be present. If a
% subcaption is not desired, just leave its contents blank,
% e.g., \subfloat[].

% An example of a floating table. Note that, for IEEE style tables, the
% \caption command should come BEFORE the table and, given that table
% captions serve much like titles, are usually capitalized except for words
% such as a, an, and, as, at, but, by, for, in, nor, of, on, or, the, to
% and up, which are usually not capitalized unless they are the first or
% last word of the caption. Table text will default to \footnotesize as
% the IEEE normally uses this smaller font for tables.
% The \label must come after \caption as always.
%
%\begin{table}[!t]
%% increase table row spacing, adjust to taste
%\renewcommand{\arraystretch}{1.3}
% if using array.sty, it might be a good idea to tweak the value of
% \extrarowheight as needed to properly center the text within the cells
%\caption{An Example of a Table}
%\label{table_example}
%\centering
%% Some packages, such as MDW tools, offer better commands for making tables
%% than the plain LaTeX2e tabular which is used here.
%\begin{tabular}{|c||c|}
%\hline
%One & Two\\
%\hline
%Three & Four\\
%\hline
%\end{tabular}
%\end{table}

% Note that the IEEE does not put floats in the very first column
% - or typically anywhere on the first page for that matter. Also,
% in-text middle ("here") positioning is typically not used, but it
% is allowed and encouraged for Computer Society conferences (but
% not Computer Society journals). Most IEEE journals/conferences use
% top floats exclusively. 
% Note that, LaTeX2e, unlike IEEE journals/conferences, places
% footnotes above bottom floats. This can be corrected via the
% \fnbelowfloat command of the stfloats package.

\subsection{State of the Art}
TODO