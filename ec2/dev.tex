\section{Development}

For this lab experience, we were given incomplete Python scripts that we had to
finish implementing. In this section, we will describe what we did to complete
said scripts.

\subsection{ICMP and Ping}
The \texttt{ping.py} script had the following functions:

\begin{itemize}
    \item \texttt{checksum}: self-explanatory
    \item \texttt{receive\_one\_ping}: uses a socket to listen for incoming packets
    \item \texttt{send\_one\_ping}: sends an ICMP echo request to a specified host
    \item \texttt{do\_one\_ping}: chains the previous two functions to send a ping
    \item \texttt{ping}: repeats pings indefinitely and displays RTT in ms
\end{itemize}

First, we completed the \texttt{receive\_one\_ping} function. The provided
template already did the job of receiving data from a socket, and we had to
unpack both the header and payload using the correct \texttt{struct} format.
Finally, we modified the function to return the time between the message and
the response. Below is a screenshot of the relevant code. Our additions are
encased in \# symbols.

\begin{figure}[htbp]
    \centering
    \includegraphics[width=1\linewidth]{img/receive_one_ping.png}
    \caption{\texttt{receive\_one\_ping}}\label{fig:receive_one_ping}
\end{figure}

We then added a couple of lines to the \texttt{do\_one\_ping} function, so it
sends a ping and then listens for the response.

\begin{figure}[htbp]
    \centering
    \includegraphics[width=1\linewidth]{img/do_one_ping.png}
    \caption{\texttt{do\_one\_ping}}\label{fig:do_one_ping}
\end{figure}

Finally, we very slightly modified the \texttt{ping} function to display the
RTT in ms. We feel it's not necessary to show a screenshot of this change.

After applying these changes, we tested the script with three different
hostnames:

\begin{itemize}
    \item google.com
    \item store.steampowered.com
    \item reachthefinals.com
\end{itemize}

The results are shown in the following three figures.

\begin{figure}[htbp]
    \centering
    \includegraphics[width=1\linewidth]{img/google.png}
    \caption{\texttt{google.com} ping}\label{fig:google}
\end{figure}

\begin{figure}[htbp]
    \centering
    \includegraphics[width=1\linewidth]{img/store_steampowered.png}
    \caption{\texttt{store.steampowered.com} ping}\label{fig:store_steampowered}
\end{figure}

\begin{figure}[htbp]
    \centering
    \includegraphics[width=1\linewidth]{img/reach_the_finals.png}
    \caption{\texttt{reachthefinals.com} ping}\label{fig:reachthefinals}
\end{figure}

We can see quite different RTTs for each host, which is expected given their
different locations (and probably different server infrastructure). The
\texttt{store.steampowered.com} ping had the lowest RTT, which is probably due
to Steam having a large number of CDNs for availability around the world.

\subsection{ICMP Traceroute}

Similarly to the previous script, we were given a template for a
\texttt{traceroute} implementation. The following functions were defined:

\begin{itemize}
    \item \texttt{checjksum}: self-explanatory
    \item \texttt{build\_packet}: builds an ICMP echo request packet
    \item \texttt{get\_route}: sends a series of packets with increasing TTL, just like the \texttt{traceroute} command does
\end{itemize}

The needed changes were quite simple. First, we copied the \texttt{checksum}
function from our ping implementation, and then we completed the packet
unpacking in the \texttt{get\_route} function. Luckily, the
\texttt{build\_packet} function was already implemented for us. Below is a
screenshot of the relevant code in \texttt{get\_route}.

\begin{figure}[htbp]
    \centering
    \includegraphics[width=1\linewidth]{img/get_route.png}
    \caption{\texttt{get\_route}}\label{fig:get_route}
\end{figure}

We then tested the script with the same three hostnames as before. When working
in Windows 11, we had to run the terminal as administrator and also disable the
firewall, otherwise, the script would not work properly. This is probably due
to the usage of raw sockets. We used a maximum TTL of 30, and a timeout of 2
seconds. The results are shown below.

\begin{figure}[htbp]
    \centering
    \includegraphics[width=1\linewidth]{img/google_traceroute.png}
    \caption{\texttt{google.com} traceroute}\label{fig:google_traceroute}
\end{figure}

\begin{figure}[htbp]
    \centering
    \includegraphics[width=1\linewidth]{img/store_steampowered_traceroute.png}
    \caption{\texttt{store.steampowered.com} traceroute}\label{fig:store_steampowered_traceroute}
\end{figure}

\begin{figure}[htbp]
    \centering
    \includegraphics[width=1\linewidth]{img/reach_the_finals_traceroute.png}
    \caption{\texttt{reachthefinals.com} traceroute}\label{fig:reachthefinals_traceroute}
\end{figure}