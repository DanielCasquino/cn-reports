\section{Theoretical Framework}

\subsection{Internet Control Message Protocol (ICMP)}
As described by~\cite{rfc792}, ICMP is used to provide feedback about
communication issues between hosts when working with the Internet Protocol
(IP). As such, it uses IP packets to send information, and contains fields such
as type and code to specify the type of message.

\subsection{\texttt{ping} Command}
The \texttt{ping} utility sends ICMP Echo Request messages (Type~8) and waits
for Echo Replies (Type~0) from the target. By timestamping the send and receive
events, it derives the round-trip time (RTT) and reports loss when no reply
arrives within a timeout. Because it relies on ICMP, \texttt{ping} requires raw
sockets or elevated privileges, and its accuracy depends on routers and hosts
honoring ICMP as defined in RFC~792~\cite{rfc792,muuss:ping}.

\subsection{\texttt{traceroute} Command}
Traceroute discovers the forward path to a destination by sending probes with
increasing Time-To-Live (TTL) values; each router decrements the TTL as
specified in RFC~791~\cite{rfc791}, and when it reaches zero the router returns
an ICMP ``Time Exceeded'' (Type~11) message. By matching each ICMP response
with the probe that triggered it, traceroute reconstructs the sequence of hops.
Paris traceroute and related tools refine this technique to avoid per-flow load
balancing artifacts, but the core mechanism still relies on TTL expiry and ICMP
error reporting~\cite{augustin2006paris}.
