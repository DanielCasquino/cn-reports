\section{State of the Art}

Early topology discovery tools such as Rocketfuel used large-scale traceroute
campaigns to infer ISP-level maps, showing that ICMP-based probing could reveal
rich router-level structure when combined with alias resolution and heuristic
cleaning~\cite{spring2002rocketfuel}. Later work like Paris traceroute refined
the method to avoid per-flow load balancers that distort paths, reducing false
links by keeping the five-tuple constant across probes~\cite{augustin2006paris}.

ICMP filtering has also been studied from a security and availability
perspective. RFC~4890 recommends selective admission of ICMPv6 control traffic
to preserve diagnostics while mitigating abuse, and measurement studies show
that overly aggressive filtering can bias latency and reachability estimates
collected via \texttt{ping}/\texttt{traceroute}~\cite{rfc4890}. For hosts that
do respond, ICMP remains a lightweight way to monitor SLA-style metrics across
geographically distributed vantage points, as demonstrated by CAIDA's active
measurement platforms and similar research testbeds.
